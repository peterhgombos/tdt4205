\documentclass{article}
\usepackage[utf8]{inputenc}
\usepackage{graphicx}
\title{Problem Set 4}
\author{Péter H. Gombos}

\begin{document}
\maketitle
\section*{Problem 1}
\subsection*{a}
The grammar is not left-recursive, as it is not recursive at all. None of the
non-terminals rewrites to itself in any way.

\subsection*{b}
The grammar is ambigious, as the parse tree of the string \emph{"zz"} can be
obtained by using the production $C \rightarrow \epsilon$ in two different
places.

\section*{Problem 2}
\subsection*{a}
All three of the grammars follow the first two conditions for an LL(1) grammar,
that is the productions starts with the same non-terminal, and that only one of
the productions produces the empty string. To fully decide if the grammars are
LL(1), we have to consider the FIRST and FOLLOW sets of the grammars. All of
the grammars have these sets equal, as the string must start with either empty
string or \emph{a}, and follows the same terminals. As the FOLLOW set will be
$\{a, \epsilon, \$\}$, and the production not leading to the empty string will
produce a string starting with \emph{a}, none of the grammars are LL(1).

\subsection*{b}
The language, that is all strings with an even number of a's, or the empty
string, is regular. A regular expression to recognise this language is $a\{2\}*$.

\section*{Problem 3}
\subsection*{a}
\begin{tabular}{ c | c | c}
matched symbol & production & string \\ \hline
S & B & B \\
B & Sb & Sb \\
S & A & Ab \\
A & xAy & xAyb \\
A & xAy & xxAyyb \\
A & z & xxzyyb
\end{tabular}

\subsection*{b}


\section*{Problem 4}
A parser with backtracking may require repeated scans over the input. Instead of
choosing the correct production to follow, it will try each possible, and only
return a failure if none of the production was correct after testing all of
them.

\section*{Problem 5}
\subsection*{a}
A hash table will give near constant time look up, and most of the operations on
a symbol table will be look ups. By using a bit more time on the insert and
delete operations, a hash table will the most efficient implementation for
symbol tables.

\subsection*{b}
Symbol tables stores information about identifiers, e.g. its lexeme (or
character string), type, position in storage.


\end{document}
