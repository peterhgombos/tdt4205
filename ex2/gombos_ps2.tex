\documentclass{article}
\usepackage[utf8]{inputenc}
\usepackage{graphicx}
\title{Problem Set 2}
\author{Péter H. Gombos}

\begin{document}
\maketitle
\section*{Problem 1}
\subsection*{a}
\begin{tabular}{ c | c | c | c | c | c | r }


NFA STATE & DFA STAE & a & b & c & d & e \\ \hline
\{ 1, 4, 5 \} &    A     & B & \emptyset & \emptyset & B & D \\ \hline
\{ 1, 2  \} & B &      B     & C & \emptyset & \emptyset & \emptyset \\ \hline
\{ 2, 3 \} & C &  \emptyset & C & E & \emptyset & E \\ \hline
\{ 5 \} & D & \emptyset & C & E & \emptyset & D \\ \hline
\{ 3, 6 \} & E &  \emptyset & \emptyset & E & \emptyset & \emptyset \\ 
\end{tabular}

As a graph this automaton will look like
\begin{center}
\includegraphics[scale=0.7]{dfa.png}
\end{center}
\subsection*{b}
The language 
\begin{math}
L = \{a^nb^n | n >= 1 \}
\end{math}
,that is the language with an equal number of a's and b's, is an example of a language 
that can not be described by a regular expression. The reason for this is that a finite 
automaton cannot keep a memory of the previous input given.

\section*{Problem 2}
\subsection*{a}
An ambigious grammar is a grammar that can produce more than one parse tree. This is a 
problem for parser because it cannot know which parse tree that is correct.

\subsection*{b}
No, as the grammar is postfix, and it is always only one way to parse any string in postfix.

\subsection*{c}
A left recursive grammar is a grammar where the left most symbol of the body is the same as 
the head of the production. Some parsers will only change the look ahead symbol after a terminal
in the body is matched. If the grammar is left recursive, it will always match the the same non 
terminal as in the head with the first occurence of the same non terminal. This will cause the
parser to loop forever.


\section*{Problem 5}
\subsection*{a}
A bottom-up parser will look at the terminals first, keeping a stack of all the characters and
tokens seen so far. If possible, it will combine the characters using the productions. Top-down
parsing, on the other hand, starts by assuming that the starting string matches the initial 
production, and tries to "solve" the string by doing the different productions.

\subsection*{b}
An LL parser is a top-down parser, while an LR parser is bottom-up.
\end{document}
